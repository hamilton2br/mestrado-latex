% abnTeX2: Modelo de Trabalho Academico (tese de doutorado, dissertacao de
% mestrado e trabalhos monograficos em geral) em conformidade com
% ABNT NBR 14724:2011: Informacao e documentacao - Trabalhos academicos -
% Apresentacao
% ------------------------------------------------------------------------

\documentclass[12pt,				% tamanho da fonte
	openright,			% capítulos começam em pág ímpar (insere página vazia caso preciso)
	oneside,			% twoside = para impressão em verso e anverso. Oposto a oneside
	a4paper,			% tamanho do papel. 
	% -- opções da classe abntex2 --
	%chapter=TITLE,		% títulos de capítulos convertidos em letras maiúsculas
	%section=TITLE,		% títulos de seções convertidos em letras maiúsculas
	%subsection=TITLE,	% títulos de subseções convertidos em letras maiúsculas
	%subsubsection=TITLE,% títulos de subsubseções convertidos em letras maiúsculas
	% -- opções do pacote babel --
	english,			% idioma adicional para hifenização
	brazil				% o último idioma é o principal do documento
	]{abntex2}

% ---
% Pacotes básicos 
% ---
\usepackage[T1]{fontenc}		% Selecao de codigos de fonte.
\usepackage[utf8]{inputenc}		% Codificacao do documento (conversão automática dos acentos)
\usepackage{lmodern}			% Usa a fonte Latin Modern			
\usepackage{cmap}				% Mapear caracteres especiais no PDF - tcc cartrac
\usepackage{lastpage}			% Usado pela Ficha catalográfica
\usepackage{indentfirst}		% Indenta o primeiro parágrafo de cada seção.
\usepackage{color}				% Controle das cores
\usepackage{graphicx}			% Inclusão de gráficos
\usepackage{microtype} 			% para melhorias de justificação
% ---	
% \usepackage{lipsum}				% para geração de dummy text
\usepackage{todonotes} 
\usepackage{url} 
% ---
% Pacotes de citações
% ---
%\usepackage[brazilian,hyperpageref]{backref}	 % Paginas com as citações na bibl
\usepackage[alf]{abntex2cite}	% Citações padrão ABNT

% --- 
% CONFIGURAÇÕES DE PACOTES
% --- 

% ---
% Configurações do pacote backref
% Usado sem a opção hyperpageref de backref
%\renewcommand{\backrefpagesname}{Citado na(s) página(s):~}
%% Texto padrão antes do número das páginas
%\renewcommand{\backref}{}
%% Define os textos da citação
%\renewcommand*{\backrefalt}[4]{
%	\ifcase #1 %
%		Nenhuma citação no texto.%
%	\or
%		Citado na página #2.%
%	\else
%		Citado #1 vezes nas páginas #2.%
%	\fi}%
% ---

% ---
% Informações de dados para CAPA e FOLHA DE ROSTO
% ---
\titulo{Modelo de plano de pesquisa - PCS5012 - 2016/1}
% baseada em redes definidas por software
\autor{Hamilton Fonte II}
\local{São Paulo}
\data{2016}
\instituicao{%
  Universidade de São Paulo
  \par
  Escola Politécnica
  \par
  Departamento de Engenharia de Computação e Sistemas Digitais}
\tipotrabalho{Plano de Pesquisa}
% O preambulo deve conter o tipo do trabalho, o objetivo, 
% o nome da instituição e a área de concentração 
% \preambulo{Tese apresentada ao Departamento de Engenharia de Computa\c{c}\~ ao e Sistemas Digitais da Escola Polit\'ecnica da Universidade de S\~ao Paulo para obten\c{c}\~ ao do T\'{i}tulo de Livre Docente.}
% ---

% ---
% Configurações de aparência do PDF final

% alterando o aspecto da cor azul
\definecolor{blue}{RGB}{41,5,195}

% informações do PDF
\makeatletter
\hypersetup{
     	%pagebackref=true,
		pdftitle={\@title}, 
		pdfauthor={\@author},
    	pdfsubject={\imprimirpreambulo},
	    pdfcreator={LaTeX with abnTeX2},
		pdfkeywords={abnt}{latex}{abntex}{abntex2}{trabalho acadêmico}, 
		colorlinks=true,       		% false: boxed links; true: colored links
    	linkcolor=blue,          	% color of internal links
    	citecolor=blue,        		% color of links to bibliography
    	filecolor=magenta,      		% color of file links
		urlcolor=blue,
		bookmarksdepth=4
}
\makeatother
% --- 

% --- 
% Espaçamentos entre linhas e parágrafos 
% --- 

% O tamanho do parágrafo é dado por:
\setlength{\parindent}{1.3cm}

% Controle do espaçamento entre um parágrafo e outro:
\setlength{\parskip}{0.2cm}  % tente também \onelineskip

% ---
% compila o indice
% ---
\makeindex
% ---

% ----
% Início do documento
% ----
\begin{document}

% Retira espaço extra obsoleto entre as frases.
\frenchspacing

% ----------------------------------------------------------
% ELEMENTOS PRÉ-TEXTUAIS
% ----------------------------------------------------------
% \pretextual

% Capa
\imprimircapa

% --- Folha de rosto
% (o * indica que haverá a ficha bibliográfica)
% \imprimirfolhaderosto*

% ---
% Inserir a ficha bibliografica

% Isto é um exemplo de Ficha Catalográfica, ou ``Dados internacionais de
% catalogação-na-publicação''. Você pode utilizar este modelo como referência. 
% Porém, provavelmente a biblioteca da sua universidade lhe fornecerá um PDF
% com a ficha catalográfica definitiva após a defesa do trabalho. Quando estiver
% com o documento, salve-o como PDF no diretório do seu projeto e substitua todo
% o conteúdo de implementação deste arquivo pelo comando abaixo:
%
% \begin{fichacatalografica}
%     \includepdf{fig_ficha_catalografica.pdf}
% \end{fichacatalografica}
%\begin{fichacatalografica}
%	\vspace*{\fill}					% Posição vertical
%	\hrule							% Linha horizontal
%	\begin{center}					% Minipage Centralizado
%	\begin{minipage}[c]{12.5cm}		% Largura
%	
%	\imprimirautor
%	
%	\hspace{0.5cm} \imprimirtitulo  / \imprimirautor. --
%	\imprimirlocal, \imprimirdata-
%	
%	\hspace{0.5cm} \pageref{LastPage} p. : \\ %il. (algumas color.) ; 30 cm.\\
%	
%	\hspace{0.5cm} \imprimirorientadorRotulo~\imprimirorientador\\
%	
%	\hspace{0.5cm}
%	\parbox[t]{\textwidth}{\imprimirtipotrabalho~--~\imprimirinstituicao,
%	\imprimirdata.}\\
%	
%	\hspace{0.5cm}
%		1. Plano de Pesquisa.
%		2. Modelo.
%		3. PCS5012.
%		I. Universidade de São Paulo. Escola Politécnica. Departamento de Engenharia
%de Computação e Sistemas Digitais
%		II. t.\\ 			
%	
%	\hspace{8.75cm} CDU 02:141:005.7\\
%	
%	\end{minipage}
%	\end{center}
%	\hrule
%\end{fichacatalografica}
% ---

\pagebreak
% ---
% Dedicatória
% ---
%\begin{dedicatoria}
%   \vspace*{\fill}
%   \flushright
%   \noindent
%   \textit{``por que?'' é a pergunta mais importante. } 
%   \vspace*{\fill}
%\end{dedicatoria}
% ---

% ---
% Agradecimentos
% ---
%\begin{agradecimentos}
%
%
%\end{agradecimentos}
% ---

% ---
% Epígrafe -- deveria ser relacionada a área do conhecimento
% ---
%\begin{epigrafe}
%
%\end{epigrafe}
% ---

% ---
% RESUMOS
% ---

% resumo em português
\setlength{\absparsep}{18pt} % ajusta o espaçamento dos parágrafos do resumo
\begin{resumo}
Esta pesquisa vai demonstrar que as abordagens para realização de análise forense de memória volátil em nuvem disponíveis hoje não lidam com as características elásticas, 
multi-inquilino e multi-jurisdição de forma satisfatória. Esperamos comprovar através da implementação de um arcabouço de coleta de informações de memória, que a solução
esta em uma mudança na abordagem da coleta e armazenamento da evidência. Tendo êxito daremos um passo para a solução dos dois maiores problemas hoje na forense em núvem, 
volume de dados e conformidade com requisitos juridicos.

% \vspace{\onelineskip}
 
%\noindent 
%\textbf{Palavras-chaves}: 

\end{resumo}

% resumo em inglês
%\begin{resumo}[Abstract]
%\begin{otherlanguage*}{english}
%
% \vspace{\onelineskip}
%  \noindent 
%  \textbf{Key-words}: 
%  
% \end{otherlanguage*}
%\end{resumo}


% ---
% inserir lista de ilustrações
% ---
%\pdfbookmark[0]{\listfigurename}{lof}
%\listoffigures*
%\cleardoublepage
% ---

% ---
% inserir lista de tabelas
% ---
%\pdfbookmark[0]{\listtablename}{lot}
%\listoftables*
%\cleardoublepage
% ---

% ---
% inserir lista de abreviaturas e siglas
% ---
%\begin{siglas}
%   \item[USP]    Universidade de São Paulo
%\end{siglas}
% ---

%% ---
%% inserir lista de símbolos
%% ---
%\begin{simbolos}
%  \item[$ \Gamma $] Letra grega Gama
%  \item[$ \Lambda $] Lambda
%  \item[$ \zeta $] Letra grega minúscula zeta
%  \item[$ \in $] Pertence
%\end{simbolos}
%% ---

% ---
% inserir o sumario
% ---
%\pdfbookmark[0]{\contentsname}{toc}
%\tableofcontents*
%\cleardoublepage
% ---

% ----------------------------------------------------------
% ELEMENTOS TEXTUAIS
% ----------------------------------------------------------
\textual

%%%%%%%%%%%%%%%%%%%%%%%%%%%%%%%%%%%%%%%%%%%%%%%%%%%%%%
\chapter{Descrição do problema de pesquisa} \label{chap:intro}
%\input{chap/01-intro}
Forense digital é um conjunto de técnicas de coleta e análise de evidências geradas por computadores que tem por objetivos, entender a sequência de eventos que permitiu que um 
ataque ocorresse, impedir que a vulnerabilidade que tornou possível o ataque seja explorada novamente e apoiar os processo jurídicos que visam submeter os culpados as punições
previstas na lei \cite{Sang2013}. 
A forense digital cresceu a partir de técnicas usadas na forense tradicional. Começou de forma artesanal, passou por uma fase de adaptação aos requisitos legais até a era atual
de ferramental avançado de coleta e análise \cite{Charters2008}.

A utilização crescente de virtualização, ferramentas online e hospedagem em nuvem \cite{Amazon2016}, esta criando dificuldades para a coleta de informações, análise e utilização 
em processos legais \cite{Sharma2012}. A funcionalidade da elasticidade de carga ofertada pelos provedores de nuvem por meio da qual infraestrutura pode ser 
alocada e desalocada dinamicamente, trouxe o problema da volatilidade dos dados nas máquinas virtuais. Com algumas ameaças que não deixam evidências em disco \cite{Rafique2013}, 
a memória de uma máquina despejada de um pool e seus recursos reciclados seria para sempre perdida e com ela evidências importantes. O simples armazenamento do conteúdo da memória
não satisfaz o requisito jurídico de se repetir o processo e conseguir os mesmos resultados. A abordagem de armazenar constantemente todas as alterações na memória não contribui 
para a solução do crescente backlog de dados que os investigadores tem para analizar \cite{Quick2014}. O ferramental forense disponível hoje esta pouco adaptado a desafios trazidos
pela nuvem \cite{Dykstra2012a}, focam em completude, resposta a incidente e poucos geram evidências aceitaveis em um processo legal \cite{Reichert2015}. A cadeia de custódia, um 
processo de coleta e armazenamento que visa garantir que a evidência não foi alterada, destruida ou manipulada por pessoas não autorizadas, é pouco abordada nas soluções existentes
hoje. A solução destes problemas passa pela confirmação das seguintes hipóteses

\begin{enumerate}
 \item \textbf{É possível conseguir o mesmo resultado da coleta mesmo se a máquina e o processo de origem não existirem mais}: Esta hipótese esta associada a necessidade jurídica de se 
 repetir o processo de coleta e conseguir os mesmos resultados em um cenário onde as evidências de um ataque estavam na memória de uma máquina virtual que não esta mais no pool.
 Como a maioria das soluções de coleta hoje se baseiam em isolamento da máquina virtual afetada ou no armazenamento de informação de log, conseguir comprovar esta hipótese nos dá
 a oportunidade de fechar uma forma de se conseguir impunidade em crimes cometidos na núvem.
 \item \textbf{Não é necessário todo o histórico de alterações da memória nem a cópia bit a bit da mesma para a análise do incidente}: Esta hipótese esta associada a realidade do crescente
 volume de informações que precisam ser analisadas pelos investigadores forenses, em 2014 a média era de 6 a 12 meses de backlog \cite{Quick2014}. A confirmação desta hipótese
 envolve a aceitação de uma mudança no paradigma de coleta de informações forenses de 'completude' para 'a coleta do necessário'.
\end{enumerate}

Esta pesquisa tenta resolver os 2 maiores problemas da forense em nuvem hoje sugerindo uma mudança no paradigma da coleta de informações de memória das máquinas. Os desafios apresentados
neste capítulo mostram que as abodagens atuais estão se esgotando e acompanhar a evolução da nuvem requer uma mudanças nos procedimentos de coleta.

\chapter{Objetivos} \label{chap:obj}
A aceitabilidade de evidências resultantes do trabalho de peritos é regulada por um conjunto de regras da lei Norte-Americana. Deste conjunto de leis derivou-se 3 requisitos que
a evidência coletada de nuvem precisa se submeter são eles, a obrigatoriedade em garantir que as provas não foram comprometidas ou alteradas durante o processo de coleta, o processo
de coleta ser capaz gerar as mesmas evidências quando executado novamente e por fim conhecer sua taxa de erro. A maioria das pesquisas relacionadas a coleta de informações tem viés
técnico e poucos lidam com os aspectos legais relacionados a coleta de informações. Sendo assim, o objetivo principal deste projeto é provar que é possível coletar de dados de 
memória de processos em máquinas na nuvem de modo que estes atendam aos 3 requisitos legais mencionados anteriormente. Para alcançar este objetivo, os seguintes subobjetivos 
precisam ser alcançados:

\begin{enumerate}
 \item Conseguir reproduzir o processo de coleta mesmo se a máquina de onde se originaram os dados tiver sido despejada e seus recursos desalocados.
 \item Conseguir realizar análise de um incidente com no máximo 10\% da informação de memória resultantes de processos de coleta existentes hoje.
 \item Conseguir realizar a coleta sem violar jurisdição e privacidade de usuários não relacionados a investigação.
 \item Conhecer a taxa de erro relacionada ao processo.
\end{enumerate}

\chapter{Método} \label{chap:metodo}
\begin{enumerate}
 \item Desenvolver uma aplicação de coleta de informações de memória em núvem de modo que se consiga relacionar a mesma a sua origem e que
o processo seja repetível. Este estudo será realizado em máquinas virtuais rodando sistemas operacionais windows e linux. A verificação do funcionamento correto será feita 
da seguinte forma:

\begin{enumerate}
 \item Realizar a primeira coleta.
 \item Despejar a máquina virtual do pool e liberar seus recursos.
 \item Recriar a máquina e executar o processo de coleta novamente.
 \item Mensurar a diferença entre as coletas através de comparação simples e guardar o percentual de diferença.
 \item Repetir os passos (b), (c) e (d) até <limite>.
\end{enumerate}

\item Desenvolver e documentar um arcabouço de armazenamento dos dados que garanta a cadeia de custódia da evidència guardando dados o suficiente para 
conseguir distinguir o momento anterior ao ataque.

\item Entrevistar analistas forenses e juizes. Apresenta-los o arcabouço de armazenamento e coletar opiniões sobre a aceitabilidade do mesmo.

\item Entrevistar analistas forenses e luizes. Apresenta-los a estratégia de armazenamento mínimo e coletar opiniões sobre a aceitabilidade do mesmo. 

\item Publicar resultados.

\item Escrita e defesa de tese.

\end{enumerate}

\chapter{Revisão bibliográfica} \label{chap:related}
A revisão bibliográfica focou nos seguintes conceitos da coleta de informações forenses em nuvem:

\begin{itemize}
 \item Acessar e coletar as informações em geral das máquinas em nuvem
 \item Conformidade com os seguintes requisitos legais: 
 \begin{itemize}
  \item reproduzir o processo com os mesmos resultados
  \item não violar privacidade de outros usuários de nuvem
  \item garantir a cadeia de custódia da evidência
 \end{itemize}
\end{itemize}

Com isso constatou-se que, referente a coleta de informações a maioria dos trabalhos mais importantes na área \cite{Reichert2015}, \cite{Poisel2013}, \cite{Dykstra2013}, \cite{George2012}
\cite{Sang2013} foca em coleta reativa, isto é, ela acontece apenas após a ameaça ser detectada. Podem acontecer por acionamento manual ou integrada a um mecanismo de detecção de ameaça. 
Em casos de memória volátil, esta forma de coleta de informação não consegue descrever como era a memória antes do ataque, o que acredito é necessário para a estratégia de 
coletar o mínimo necessário para realizar a investigação e apoiar o processo jurídico. A única proposta encontrada que leva esta necessidade em consideração é \cite{Dezfouli2012} mas
propoe que o dado seja armazenado no próprio dispositivo o que pode levar a perda das informações caso a máquina seja despejada do pool e seus recursos liberados.

Ainda na coleta de informações, \cite{Reichert2015}, \cite{George2012} sugerem a abordagem de forense ao vivo onde os dados são coletados com o sistema rodando enquanto 
\cite{Poisel2013}, \cite{Dykstra2013}, \cite{Sang2013} adotam a estratégia de isolar e/ou pausar a máquina virtual e em seguida disparar o processo de coleta. Nos dois casos, 
nenhum dos autores cobre o cenário onde a máquina é despejada do pool e os recursos são liberados. Analisando cuidadosamente as propostas de \cite{Poisel2013}, parece que é possível
cobrir o cenário mencionado anteriormente mas ele não dá detalhes da implementação o suficiente para termos certeza.

Referente aos requisitos legais, nenhuma das propostas consegue reproduzir os mesmos resultados ao repetir o processo no cenário em que uma máquina é despejada da nuvem e seus 
recursos liberados. Analisando cuidadosamente a proposta de \cite{George2012} parece que é possível mas o autor não dá detalhes de implementação suficientes para termos certeza.

No armazenamento das informações coletadas, \cite{Reichert2015}, \cite{George2012}, \cite{Poisel2013} e \cite{Dykstra2013} usam estratégia de armazenamento fora da nuvem para 
resolver o problema de violação de privacidade de outros usuários no processo da coleta. \cite{Sang2013} e um caso específico de \cite{George2012} dependem de cooperação do 
provedor de nuvem para conseguir as informações necessárias à investigação.

Na garantia da cadeia de custódia apenas \cite{Sang2013} faz alguma tentativa de resolver a questão porem cuida apenas da garantia que a evidência não foi destruida ou alterada a
traves de cálculo de hashing mas não explica como impede o acesso não autorizado à evidência. As propostas dos outros autores estão focadas apenas no aspecto técnico da coleta.

\chapter{Resultados esperados} \label{chap:result}
O primeiro resultado esperado é um arcabouço que consiga coletar informações de memória de processos de máquinas na nuvem e armazená-las em um local físico conhecido.
Este armazenamento deve garantir a cadeia de custódia e o volume de informação armazenada será o suficiente para diferenciar o sistema antes e depois que a ameaça foi 
detectada. A utilização deste arcabouço sob as mesmas condições deve gerar um conjunto idêntico de evidências. Com isso, será possível fechar uma das formas de impunidade 
para os atos ilícitos em nuvem e contribuir para a diminuição do backlog de investigação.

\bibliography{abntex2-modelo-references}

% ----------------------------------------------------------
% ELEMENTOS PÓS-TEXTUAIS
% ----------------------------------------------------------
\postextual

% ----------------------------------------------------------
% Referências bibliográficas
% ----------------------------------------------------------
%  \bibliography{•}

%---------------------------------------------------------------------
% INDICE REMISSIVO
%---------------------------------------------------------------------
\begin{comment}
\printindex
\end{comment}

%%%%%%%%%%%%%%%%%%%%%%%%%%%
\end{document}
