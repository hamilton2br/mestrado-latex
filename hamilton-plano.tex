% abnTeX2: Modelo de Trabalho Academico (tese de doutorado, dissertacao de
% mestrado e trabalhos monograficos em geral) em conformidade com
% ABNT NBR 14724:2011: Informacao e documentacao - Trabalhos academicos -
% Apresentacao
% ------------------------------------------------------------------------

\documentclass[12pt,				% tamanho da fonte
	openright,			% capítulos começam em pág ímpar (insere página vazia caso preciso)
	oneside,			% twoside = para impressão em verso e anverso. Oposto a oneside
	a4paper,			% tamanho do papel. 
	% -- opções da classe abntex2 --
	%chapter=TITLE,		% títulos de capítulos convertidos em letras maiúsculas
	%section=TITLE,		% títulos de seções convertidos em letras maiúsculas
	%subsection=TITLE,	% títulos de subseções convertidos em letras maiúsculas
	%subsubsection=TITLE,% títulos de subsubseções convertidos em letras maiúsculas
	% -- opções do pacote babel --
	english,			% idioma adicional para hifenização
	brazil				% o último idioma é o principal do documento
	]{abntex2}

% ---
% Pacotes básicos 
% ---
\usepackage[T1]{fontenc}		% Selecao de codigos de fonte.
\usepackage[utf8]{inputenc}		% Codificacao do documento (conversão automática dos acentos)
\usepackage{lmodern}			% Usa a fonte Latin Modern			
\usepackage{cmap}				% Mapear caracteres especiais no PDF - tcc cartrac
\usepackage{lastpage}			% Usado pela Ficha catalográfica
\usepackage{indentfirst}		% Indenta o primeiro parágrafo de cada seção.
\usepackage{color}				% Controle das cores
\usepackage{graphicx}			% Inclusão de gráficos
\usepackage{microtype} 			% para melhorias de justificação
% ---	
% \usepackage{lipsum}				% para geração de dummy text
\usepackage{todonotes} 
\usepackage{url} 
% ---
% Pacotes de citações
% ---
%\usepackage[brazilian,hyperpageref]{backref}	 % Paginas com as citações na bibl
\usepackage[alf]{abntex2cite}	% Citações padrão ABNT

% --- 
% CONFIGURAÇÕES DE PACOTES
% --- 

% ---
% Configurações do pacote backref
% Usado sem a opção hyperpageref de backref
%\renewcommand{\backrefpagesname}{Citado na(s) página(s):~}
%% Texto padrão antes do número das páginas
%\renewcommand{\backref}{}
%% Define os textos da citação
%\renewcommand*{\backrefalt}[4]{
%	\ifcase #1 %
%		Nenhuma citação no texto.%
%	\or
%		Citado na página #2.%
%	\else
%		Citado #1 vezes nas páginas #2.%
%	\fi}%
% ---

% ---
% Informações de dados para CAPA e FOLHA DE ROSTO
% ---
\titulo{Modelo de plano de pesquisa - PCS5012 - 2016/1}
% baseada em redes definidas por software
\autor{Hamilton Fonte II}
\local{São Paulo}
\data{2016}
\instituicao{%
  Universidade de São Paulo
  \par
  Escola Politécnica
  \par
  Departamento de Engenharia de Computação e Sistemas Digitais}
\tipotrabalho{Plano de Pesquisa}
% O preambulo deve conter o tipo do trabalho, o objetivo, 
% o nome da instituição e a área de concentração 
% \preambulo{Tese apresentada ao Departamento de Engenharia de Computa\c{c}\~ ao e Sistemas Digitais da Escola Polit\'ecnica da Universidade de S\~ao Paulo para obten\c{c}\~ ao do T\'{i}tulo de Livre Docente.}
% ---

% ---
% Configurações de aparência do PDF final

% alterando o aspecto da cor azul
\definecolor{blue}{RGB}{41,5,195}

% informações do PDF
\makeatletter
\hypersetup{
     	%pagebackref=true,
		pdftitle={\@title}, 
		pdfauthor={\@author},
    	pdfsubject={\imprimirpreambulo},
	    pdfcreator={LaTeX with abnTeX2},
		pdfkeywords={abnt}{latex}{abntex}{abntex2}{trabalho acadêmico}, 
		colorlinks=true,       		% false: boxed links; true: colored links
    	linkcolor=blue,          	% color of internal links
    	citecolor=blue,        		% color of links to bibliography
    	filecolor=magenta,      		% color of file links
		urlcolor=blue,
		bookmarksdepth=4
}
\makeatother
% --- 

% --- 
% Espaçamentos entre linhas e parágrafos 
% --- 

% O tamanho do parágrafo é dado por:
\setlength{\parindent}{1.3cm}

% Controle do espaçamento entre um parágrafo e outro:
\setlength{\parskip}{0.2cm}  % tente também \onelineskip

% ---
% compila o indice
% ---
\makeindex
% ---

% ----
% Início do documento
% ----
\begin{document}

% Retira espaço extra obsoleto entre as frases.
\frenchspacing

% ----------------------------------------------------------
% ELEMENTOS PRÉ-TEXTUAIS
% ----------------------------------------------------------
% \pretextual

% Capa
\imprimircapa

% --- Folha de rosto
% (o * indica que haverá a ficha bibliográfica)
% \imprimirfolhaderosto*

% ---
% Inserir a ficha bibliografica

% Isto é um exemplo de Ficha Catalográfica, ou ``Dados internacionais de
% catalogação-na-publicação''. Você pode utilizar este modelo como referência. 
% Porém, provavelmente a biblioteca da sua universidade lhe fornecerá um PDF
% com a ficha catalográfica definitiva após a defesa do trabalho. Quando estiver
% com o documento, salve-o como PDF no diretório do seu projeto e substitua todo
% o conteúdo de implementação deste arquivo pelo comando abaixo:
%
% \begin{fichacatalografica}
%     \includepdf{fig_ficha_catalografica.pdf}
% \end{fichacatalografica}
%\begin{fichacatalografica}
%	\vspace*{\fill}					% Posição vertical
%	\hrule							% Linha horizontal
%	\begin{center}					% Minipage Centralizado
%	\begin{minipage}[c]{12.5cm}		% Largura
%	
%	\imprimirautor
%	
%	\hspace{0.5cm} \imprimirtitulo  / \imprimirautor. --
%	\imprimirlocal, \imprimirdata-
%	
%	\hspace{0.5cm} \pageref{LastPage} p. : \\ %il. (algumas color.) ; 30 cm.\\
%	
%	\hspace{0.5cm} \imprimirorientadorRotulo~\imprimirorientador\\
%	
%	\hspace{0.5cm}
%	\parbox[t]{\textwidth}{\imprimirtipotrabalho~--~\imprimirinstituicao,
%	\imprimirdata.}\\
%	
%	\hspace{0.5cm}
%		1. Plano de Pesquisa.
%		2. Modelo.
%		3. PCS5012.
%		I. Universidade de São Paulo. Escola Politécnica. Departamento de Engenharia
%de Computação e Sistemas Digitais
%		II. t.\\ 			
%	
%	\hspace{8.75cm} CDU 02:141:005.7\\
%	
%	\end{minipage}
%	\end{center}
%	\hrule
%\end{fichacatalografica}
% ---

\pagebreak
% ---
% Dedicatória
% ---
%\begin{dedicatoria}
%   \vspace*{\fill}
%   \flushright
%   \noindent
%   \textit{``por que?'' é a pergunta mais importante. } 
%   \vspace*{\fill}
%\end{dedicatoria}
% ---

% ---
% Agradecimentos
% ---
%\begin{agradecimentos}
%
%
%\end{agradecimentos}
% ---

% ---
% Epígrafe -- deveria ser relacionada a área do conhecimento
% ---
%\begin{epigrafe}
%
%\end{epigrafe}
% ---

% ---
% RESUMOS
% ---

% resumo em português
\setlength{\absparsep}{18pt} % ajusta o espaçamento dos parágrafos do resumo
\begin{resumo}
O objetivo é coletar dados de memória de processos em máquinas em núvem de modo que os mesmos sejam aceitos em um processo legal. Através de um método que visa \\
associar os dados relacionados a máquina de onde eles saíram, de modo a ajudar a diminuir a quantidade de dados a ser analisada propondo uma mudança no paradigma \\
da coleta assim fechando uma rota de fuga para criminosos escaparem pela via legal.

Segundo a norma ABNT, o resumo deve ressaltar o  objetivo, o método, os resultados e as conclusões do documento.

% \vspace{\onelineskip}
 
%\noindent 
%\textbf{Palavras-chaves}: 

\end{resumo}

% resumo em inglês
%\begin{resumo}[Abstract]
%\begin{otherlanguage*}{english}
%
% \vspace{\onelineskip}
%  \noindent 
%  \textbf{Key-words}: 
%  
% \end{otherlanguage*}
%\end{resumo}


% ---
% inserir lista de ilustrações
% ---
%\pdfbookmark[0]{\listfigurename}{lof}
%\listoffigures*
%\cleardoublepage
% ---

% ---
% inserir lista de tabelas
% ---
%\pdfbookmark[0]{\listtablename}{lot}
%\listoftables*
%\cleardoublepage
% ---

% ---
% inserir lista de abreviaturas e siglas
% ---
%\begin{siglas}
%   \item[USP]    Universidade de São Paulo
%\end{siglas}
% ---

%% ---
%% inserir lista de símbolos
%% ---
%\begin{simbolos}
%  \item[$ \Gamma $] Letra grega Gama
%  \item[$ \Lambda $] Lambda
%  \item[$ \zeta $] Letra grega minúscula zeta
%  \item[$ \in $] Pertence
%\end{simbolos}
%% ---

% ---
% inserir o sumario
% ---
%\pdfbookmark[0]{\contentsname}{toc}
%\tableofcontents*
%\cleardoublepage
% ---

% ----------------------------------------------------------
% ELEMENTOS TEXTUAIS
% ----------------------------------------------------------
\textual

%%%%%%%%%%%%%%%%%%%%%%%%%%%%%%%%%%%%%%%%%%%%%%%%%%%%%%
\chapter{Descrição do problema de pesquisa} \label{chap:intro}
%\input{chap/01-intro}
Forense digital é um conjunto de técnicas de coleta e análise de evidências geradas por computadores que tem por objetivos, entender a sequência de eventos que permitiu que um 
ataque fosse perpetrado, impedir que a vulnerabilidade que tornou possível um ataque seja explorada novamente e apoiar um processo legal \cite{Sang2013}. A forense digital 
cresceu a partir de técnicas usadas na forense tradicional. Começou de forma ad-hoc, passou por uma fase de adaptação aos requisitos legais até a era atual de ferramental
avançado de coleta e análise \cite{Charters2008}.

A utilização crescente de virtualização, ferramentas online e hospredagem em núvem \cite{Amazon2016}, esta inviabilizando algumas práticas forenses \cite{Sharma2012}. 
Especificamente a funcionalidade da elasticidade de carga ofertada pelos provedores de nuvem por meio da qual infraestrutura pode ser alocada e desalocada dinamicamente, 
trouxe o problema da volatilidade dos dados nas máquinas virtuais. Com algumas ameaças que não deixam evidências em disco \cite{Rafique2013}, a memória de uma máquina 
despejada de um pool e seus recursos reciclados seria para sempre perdida e com ela evidências importantes. O simples armazenamento do conteúdo da memória não satisfaz 
o princípio de Daubert. A abordagem de armazenar constantemente todas as alterações na memória não contribui para a solução do crescente backlog de dados que os investigadores 
tem para analizar \cite{Quick2014}. O ferramental forense disponível hoje esta pouco adaptado a desafios trazidos pela nuvem \cite{Dykstra2012a}, focam em completude,
resposta a incidente e raros geram evidências aceitaveis em um processo legal \cite{Reichert2015}. \\

A solução deste problema passa pela confirmação das seguintes hipóteses:

\begin{enumerate}
 \item É possível conseguir o mesmo resultado da coleta mesmo se a máquina e o processo de origem não existirem mais.
 \item Não é necessário todo os histórico de alterações da memória nem a cópia bit a bit da mesma para a análise do incidente.
\end{enumerate}

Essa pesquisa se justifica pela necessidade da forense em se adaptar a computação em nuvem. Não ser capaz de levantar evidências impede investigações, prisão de criminosos e 
correção de vulnerabilidades. Tal cenário é um convite a criminalidade e garantia de impunidade.

\chapter{Objetivos} \label{chap:obj}
O princípio de Daubert é uma norma da lei Norte-Americana que versa sobre a admissibilidade de evidências oriundas do trabalho de peritos em um processo legal. Pertinentes a
esta pesquisa estão a obrigatoriedade em garantir que as provas não foram comprometidas ou alteradas durante o processo de coleta, a reprodutibilidade do processo e
por fim conhecer sua taxa de erro. Sendo assim, o objetivo principal deste projeto é provar que é possível coletar de dados de memória de processos em 
máquinas na nuvem de modo que estes se submetam com sucesso a parte cabível do Princípio de Daubert. Para alcançar este objetivo, os seguintes subobjetivos precisam ser alcançados:

\begin{enumerate}
 \item Conseguir reproduzir o processo de coleta mesmo se a máquina de onde se originaram os dados tiver sido despejada e seus recursos desalocados.
 \item Conseguir realizar análise de um incidente com no máximo 10\% da informação de memória resultantes de processos de coleta existentes hoje.
 \item Conseguir realizar a coleta sem violar jurisdição e privacidade de usuários não relacionados a investigação.
 \item Conhecer a taxa de erro relacionada ao processo.
\end{enumerate}

\chapter{Método} \label{chap:metodo}
Através de conteinerização, janela de dados e coleta fora do sistema de nuvem. Apresentar a evidência para um devogado, sei lá

\chapter{Revisão bibliográfica} \label{chap:related}
\begin{itemize}
 \item \textbf{Automated Forensic Data Acquisition in the Cloud \cite{Reichert2015} }: Propõe um modelo de coleta de dados de máquinas virtuais acionado por um sistema de 
 detecção de ameaça. Usa o Google Rapid Response para salvar as informações coletadas fora da núvem de forma a driblar os problemas de multi-jurisdição e multi-inquilino.\\
 O modelo proposto não cobre o cenário em que uma máquina é despejada do pool e seus recursos liberados após a detecção da ameaça e não explica como vai manter a cadeia de 
 custódia da evidência.
 \item \textbf{Digital forensics framework for a cloud environment \cite{George2012} }: Framework para coleta de dados de máquinas virtuais e associa-las aos usuários da 
 núvem. Usa a estratégia de armazenar tudo para evitar perda de informação decorrente do despejo de uma máquina do pool e desalocação de seus recursos. Armazena os dados coletados
 na própria nuvem.\\
 O framework proposto, apesar de cobrir o cenário em que uma máquina é despejada do pool e os recursos liberados, não dá detalhes o suficiente para demonstrar que consegue 
 reproduzir o processo e obter o mesmo resultado. É focado no usuário e não no hardware e leva algum tempo para identificar os usuários envolvidos. Usando a própria núvem para 
 armazenar os dados coletados ele depende de cooperação do provedor de núvem para obter tais dados.
 \item \textbf{Evidence and cloud computing the virtual machine introspection approach \cite{Poisel2013} }: Descreve o processo de coleta de informações de máquinas
 em núvem através da técnica de introspecção em máquina virtual, onde se acessa os dados das máquinas virtuais através do hypervisor. Propõe que o processo seja disparado 
 por demanda atrelado a um sistema de detecção de ameaça ou intervenção humana.\\
 A técnica descrita cobre apenas o processo de coleta de informações. No que tange as informações de memória, como os endereços de memória são os do host, estes precisam ser traduzidos
 para que a análise forense seja feita. Como a abordagem não tem conhecimento do que esta rodando dentro da máquina virtual precisa de uma copia bit a bit da evidência. Embora pareça
 possível, não descreve como lida com o cenário onde uma máquina é despejada do pool e os recursos liberados.
 \item \textbf{Design and implementation of FROST: FoRensic tools for Open STack \cite{Dykstra2013} }: Framework para coleta de dados de máquinas virtuais através da API do 
 hypervisor, usa técnicas reconhecidas atualmente para coleta e armazenamento dos dados. Isola a máquina virtual afetada do pool original para realização da coleta. Precisa 
 ser acionado quando uma ameaça é detectada.\\
 O framework proposto não cobre o cenário em que uma máquina é despejada do pool e os recursos liberados após a detecção da ameaça. Não conhece o que esta rodando na máquina 
 virtual, tem acesso apenas as informações que são disponibilizadas pela API do hypervisor e não explica como vai manter a cadeia de custódia.
 \item \textbf{A log based approach to make digital forensics easier on cloud computing \cite{Sang2013} }: Método que sugere salvar a informação coletada fora da núvem de modo a driblar
 os problemas de multi-inquilinato e multi-jurisdição, usa um mecanismo de hash para garantir a autenticidade e integridade da informação mas não dá detalhes da implementação. 
 Dá exemplos de como funcionaria em PaaS e SaaS.\\
 Segundo o próprio autor, o método não funciona em IaaS. Precisa da cooperação do provedor de nuvem pois depende que este último decida quais informações serão adicionadas ao log.
 O método não é aplicável a coleta de informações de memória.
\item \textbf{Volatile memory acquisition using backup for forensic investigation \cite{Dezfouli2012} }: Técnica que sugere a utilização da própria máquina
como repositório de informações de memória volátil, para manter a utilização de espaço ao mínimo sugere manter apenas o último estado conhecido de cada processo.\\
A técnica não cobre o cenário em que uma máquina é despejada do pool e seus recursos são reutilizados, de fato, esta técnica provocaria a perda de todas as evidências caso a 
liberação de recursos da máquina ocorra.
\end{itemize}

\chapter{Resultados esperados} \label{chap:result}
Fechar uma das portas de fuga para criminosos na núvem hoje.Fornecer mais um mecanismo de coleta de dados para investigação. Propor uma mudança de paradigma da coleta de modo a \\
diminuir a quantidade de informação que precisa de análise.

\bibliography{abntex2-modelo-references}

% ----------------------------------------------------------
% ELEMENTOS PÓS-TEXTUAIS
% ----------------------------------------------------------
\postextual

% ----------------------------------------------------------
% Referências bibliográficas
% ----------------------------------------------------------
%  \bibliography{•}

%---------------------------------------------------------------------
% INDICE REMISSIVO
%---------------------------------------------------------------------
\begin{comment}
\printindex
\end{comment}

%%%%%%%%%%%%%%%%%%%%%%%%%%%
\end{document}
