
\documentclass[
	% -- opções da classe memoir --
	12pt,				% tamanho da fonte
	openright,			% capítulos começam em pág ímpar (insere página vazia caso preciso)
	twoside,			% para impressão em verso e anverso. Oposto a oneside
	a4paper,			% tamanho do papel. 
	% -- opções da classe abntex2 --
	%chapter=TITLE,		% títulos de capítulos convertidos em letras maiúsculas
	%section=TITLE,		% títulos de seções convertidos em letras maiúsculas
	%subsection=TITLE,	% títulos de subseções convertidos em letras maiúsculas
	%subsubsection=TITLE,% títulos de subsubseções convertidos em letras maiúsculas
	% -- opções do pacote babel --
	english,			% idioma adicional para hifenização
	french,				% idioma adicional para hifenização
	spanish,			% idioma adicional para hifenização
	brazil,				% o último idioma é o principal do documento
	]{abntex2}

% ---
% PACOTES
% ---

% ---
% Pacotes fundamentais 
% ---
\usepackage{lmodern}			% Usa a fonte Latin Modern
\usepackage[T1]{fontenc}		% Selecao de codigos de fonte.
\usepackage[utf8]{inputenc}		% Codificacao do documento (conversão automática dos acentos)
\usepackage{indentfirst}		% Indenta o primeiro parágrafo de cada 
\usepackage{color}				% Controle das cores
\usepackage{graphicx}			% Inclusão de gráficos
\usepackage{microtype} 			% para melhorias de justificação
% ---

% ---
% Pacotes adicionais, usados apenas no âmbito do Modelo Canônico do abnteX2
% ---
\usepackage{lipsum}				% para geração de dummy text
% ---

% ---
% Pacotes de citações
% ---
\usepackage[brazilian,hyperpageref]{backref}	 % Paginas com as citações na bibl
\usepackage[alf]{abntex2cite}	% Citações padrão ABNT

% --- 
% CONFIGURAÇÕES DE PACOTES
% --- 

% ---
% Configurações do pacote backref
% Usado sem a opção hyperpageref de backref
\renewcommand{\backrefpagesname}{Citado na(s) página(s):~}
% Texto padrão antes do número das páginas
\renewcommand{\backref}{}
% Define os textos da citação
\renewcommand*{\backrefalt}[4]{
	\ifcase #1 %
		Nenhuma citação no texto.%
	\or
		Citado na página #2.%
	\else
		Citado #1 vezes nas páginas #2.%
	\fi}%
% ---

% ---
% Informações de dados para CAPA e FOLHA DE ROSTO
% ---
\titulo{Coletando dados de memória de uma máquina em nuvem para análise forense}
\autor{Hamilton Fonte II}
\local{São Paulo, Brasil}
\data{2016, v-0.1}
\instituicao{%
  Universidade de São Paulo -- USP
  \par
  Escola Politécnica - Engenharia de Computação
  \par
  Programa de Pós-Graduação em Engenharia Elétrica}
\tipotrabalho{Plano de Pesquisa de Pós-Graduação}
% O preambulo deve conter o tipo do trabalho, o objetivo, 
% o nome da instituição e a área de concentração 
\preambulo{Projeto de pesquisa para a disciplina Metodolodia de Pesquisa 
Científica em Engenharia de Computação.}
% ---

% ---
% Configurações de aparência do PDF final

% alterando o aspecto da cor azul
\definecolor{blue}{RGB}{41,5,195}

% informações do PDF
\makeatletter
\hypersetup{
     	%pagebackref=true,
		pdftitle={\@title}, 
		pdfauthor={\@author},
    	pdfsubject={\imprimirpreambulo},
	    pdfcreator={LaTeX with abnTeX2},
		pdfkeywords={abnt}{latex}{abntex}{abntex2}{projeto de pesquisa}, 
		colorlinks=true,       		% false: boxed links; true: colored links
    	linkcolor=blue,          	% color of internal links
    	citecolor=blue,        		% color of links to bibliography
    	filecolor=magenta,      		% color of file links
		urlcolor=blue,
		bookmarksdepth=4
}
\makeatother
% --- 

% --- 
% Espaçamentos entre linhas e parágrafos 
% --- 

% O tamanho do parágrafo é dado por:
\setlength{\parindent}{1.3cm}

% Controle do espaçamento entre um parágrafo e outro:
\setlength{\parskip}{0.2cm}  % tente também \onelineskip

% ---
% compila o indice
% ---
\makeindex
% ---

% ----
% Início do documento
% ----
\begin{document}

% Retira espaço extra obsoleto entre as frases.
\frenchspacing 

% ----------------------------------------------------------
% ELEMENTOS PRÉ-TEXTUAIS
% ----------------------------------------------------------
% \pretextual

% ---
% Capa
% ---
\imprimircapa
% ---

% ---
% Folha de rosto
% ---
\imprimirfolhaderosto
% ---

% ---
% NOTA DA ABNT NBR 15287:2011, p. 4:
%  ``Se exigido pela entidade, apresentar os dados curriculares do autor em
%     folha ou página distinta após a folha de rosto.''
% ---

% ---
% inserir lista de ilustrações
% ---
\pdfbookmark[0]{\listfigurename}{lof}
\listoffigures*
\cleardoublepage

% ---

% ---
% inserir lista de tabelas
% ---
\pdfbookmark[0]{\listtablename}{lot}
\listoftables*
\cleardoublepage

% ---

% ---
% inserir lista de abreviaturas e siglas
% ---
\begin{siglas}
  \item[Fig.] Area of the $i^{th}$ component
  \item[456] Isto é um número
  \item[123] Isto é outro número
  \item[lauro cesar] este é o meu nome
\end{siglas}
% ---

% ---
% inserir lista de símbolos
% ---
\begin{simbolos}
  \item[$ \Gamma $] Letra grega Gama
  \item[$ \Lambda $] Lambda
  \item[$ \zeta $] Letra grega minúscula zeta
  \item[$ \in $] Pertence
\end{simbolos}
% ---

% ---
% inserir o sumario
% ---
\tableofcontents
\cleardoublepage
% ---


% ----------------------------------------------------------
% ELEMENTOS TEXTUAIS
% ----------------------------------------------------------
\textual

% ----------------------------------------------------------
% Introdução
% ----------------------------------------------------------
\chapter{Introdução}

Os crescentes avanços tecnilógicos que têm ajudado a vida do ser humano também trouxeram diversos novos tipos de crimes relacionados a tecnologia da informação. 
Segundo levantamento de 2015 do Colégio Notarial Brasileiro houve um aumento de 87\% no número de atas que comprovam abusos e crimes virtuais (citação).
Quando do evento de um crime, há a necessidade de preservar as evidências de modo que apoiem processos investigativos e legais mas também para que possamos compreender a sequência
de eventos envolvidos no ataque de modo que possamos evitar que estas vulnerabilidades sejam exploradas novamente.

\par

Ciência forense digital é uma ferramenta importante para a solução de crimes cometidos com um computador pela sua capacidade de reconstruir a cena do crime a partir das evidências
digitais deixadas por um ataque. Com o crescente uso de ferramentas online, virtualização e hospedagem em nuvem, o volume de dados gerado por esses sistemas que já era uma preocupação
da comunidade forense em 2008 (citação), hoje começou a inviabilizar algumas de suas práticas. (citação)

\par

Coma funcionalidade da elasticidade de carga ofertada pelos provedores de nuvem, por meio da qual infraestrutura pode ser alocada e desalocada dinamicamente, veio à tona o problema
da volatilidade dos dados nas máquinas virtuais. Com algumas ameaças que não deixam evidências em disco(citação), a memória de uma máquina despejada de um pool seria para sempre
perdida. Com a máquina mais poderosa hoje ofertada pela Amazon tendo 240Gb de memória (citação), o ferramental forense disponível hoje esta pouco adaptado a desafios trazidos 
pela núvem (citação).

% ----------------------------------------------------------
% Capitulo de justificativa 
% ----------------------------------------------------------
\chapter{Justificativa}



Falar das dificuldades para se realizar forense em sistemas em nuvem, do 
crescimentos dos crimes digitais, todos os crimes tem alguma parcela de 
evidência digital, da adoção da nuvem mundo a fora, da dificuldade das 
problemas de jurisdicao por causa da de implementação da nuvem do aspecto multi 
inquilino das implementacoes em nuvem. As características específicas da memória 
para este casos associadas ao caráter elástico das soluções em nuvem. Das 
tentativas em outros campos de tentar resolver essas questões técnicas. A 
necessidade de se submeter a regras de cunho legal para as provas geradas pela 
forense.

% ----------------------------------------------------------
% Capitulo de Objetivos 
% ----------------------------------------------------------
\chapter{Objetivos}

Declarar e explicar as 3 hipõteses do trabalho, associa-los ao objetivo principal e os 3 sub-objetivos

% ----------------------------------------------------------
% Plano de Trabalho 
% ----------------------------------------------------------
\chapter{Plano de Trabalho}

Não faço ideia de como faremos isso mas estou muito interessado em saber... :)

% ----------------------------------------------------------
% Material e métodos
% ----------------------------------------------------------
\chapter{Material e Métodos}

Implementação de virtualização de mercado.
rede
laboratorio

% ----------------------------------------------------------
% Análise dos resultados
% ----------------------------------------------------------
\chapter{Análise dos Resultados}

Nao faço a mais remota ideia de como fazer isso

% ---
% Finaliza a parte no bookmark do PDF
% para que se inicie o bookmark na raiz
% e adiciona espaço de parte no Sumário
% ---
\phantompart

% ---
% Conclusão
% ---
\chapter{Considerações finais}

considerei finalmente que estamos gastando mais tempo formatando o documento do 
que realmente escrevendo seu conteúdo.

% ----------------------------------------------------------
% Referências bibliográficas
% ----------------------------------------------------------
\bibliography{abntex2-modelo-references}

%---------------------------------------------------------------------
% INDICE REMISSIVO
%---------------------------------------------------------------------

\phantompart

\printindex


\end{document}