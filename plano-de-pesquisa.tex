
\documentclass[
	% -- opções da classe memoir --
	12pt,				% tamanho da fonte
	openright,			% capítulos começam em pág ímpar (insere página vazia caso preciso)
	oneside,			% para impressão em verso e anverso. Oposto a oneside
	a4paper,			% tamanho do papel. 
	% -- opções da classe abntex2 --
	%chapter=TITLE,		% títulos de capítulos convertidos em letras maiúsculas
	%section=TITLE,		% títulos de seções convertidos em letras maiúsculas
	%subsection=TITLE,	% títulos de subseções convertidos em letras maiúsculas
	%subsubsection=TITLE,% títulos de subsubseções convertidos em letras maiúsculas
	% -- opções do pacote babel --
	english,			% idioma adicional para hifenização
	french,				% idioma adicional para hifenização
	spanish,			% idioma adicional para hifenização
	brazil,				% o último idioma é o principal do documento
	]{abntex2}

% ---
% PACOTES
% ---

% ---
% Pacotes fundamentais 
% ---
\usepackage{lmodern}			% Usa a fonte Latin Modern
\usepackage[T1]{fontenc}		% Selecao de codigos de fonte.
\usepackage[utf8]{inputenc}		% Codificacao do documento (conversão automática dos acentos)
\usepackage{indentfirst}		% Indenta o primeiro parágrafo de cada 
\usepackage{color}				% Controle das cores
\usepackage{graphicx}			% Inclusão de gráficos
\usepackage{microtype} 			% para melhorias de justificação
% ---

% ---
% Pacotes adicionais, usados apenas no âmbito do Modelo Canônico do abnteX2
% ---
\usepackage{lipsum}				% para geração de dummy text
% ---

% ---
% Pacotes de citações
% ---
\usepackage[brazilian,hyperpageref]{backref}	 % Paginas com as citações na bibl
\usepackage[alf]{abntex2cite}	% Citações padrão ABNT

% --- 
% CONFIGURAÇÕES DE PACOTES
% --- 

% ---
% Configurações do pacote backref
% Usado sem a opção hyperpageref de backref
\renewcommand{\backrefpagesname}{Citado na(s) página(s):~}
% Texto padrão antes do número das páginas
\renewcommand{\backref}{}
% Define os textos da citação
\renewcommand*{\backrefalt}[4]{
	\ifcase #1 %
		Nenhuma citação no texto.%
	\or
		Citado na página #2.%
	\else
		Citado #1 vezes nas páginas #2.%
	\fi}%
% ---

% ---
% Informações de dados para CAPA e FOLHA DE ROSTO
% ---
\titulo{Coletando dados de memória de uma máquina em nuvem para análise forense}
\autor{Hamilton Fonte II}
\local{São Paulo, Brasil}
\data{2016, v-0.1}
\instituicao{%
  Universidade de São Paulo -- USP
  \par
  Escola Politécnica - Engenharia de Computação
  \par
  Programa de Pós-Graduação em Engenharia Elétrica}
\tipotrabalho{Plano de Pesquisa de Pós-Graduação}
% O preambulo deve conter o tipo do trabalho, o objetivo, 
% o nome da instituição e a área de concentração 
\preambulo{Projeto de pesquisa para a disciplina Metodolodia de Pesquisa 
Científica em Engenharia de Computação.}
% ---

% ---
% Configurações de aparência do PDF final

% alterando o aspecto da cor azul
\definecolor{blue}{RGB}{41,5,195}

% informações do PDF
\makeatletter
\hypersetup{
     	%pagebackref=true,
		pdftitle={\@title}, 
		pdfauthor={\@author},
    	pdfsubject={\imprimirpreambulo},
	    pdfcreator={LaTeX with abnTeX2},
		pdfkeywords={abnt}{latex}{abntex}{abntex2}{projeto de pesquisa}, 
		colorlinks=true,       		% false: boxed links; true: colored links
    	linkcolor=blue,          	% color of internal links
    	citecolor=blue,        		% color of links to bibliography
    	filecolor=magenta,      		% color of file links
		urlcolor=blue,
		bookmarksdepth=4
}
\makeatother
% --- 

% --- 
% Espaçamentos entre linhas e parágrafos 
% --- 

% O tamanho do parágrafo é dado por:
\setlength{\parindent}{1.3cm}

% Controle do espaçamento entre um parágrafo e outro:
\setlength{\parskip}{0.2cm}  % tente também \onelineskip

% ---
% compila o indice
% ---
\makeindex
% ---

% ----
% Início do documento
% ----
\begin{document}

% Retira espaço extra obsoleto entre as frases.
\frenchspacing 

% ----------------------------------------------------------
% ELEMENTOS PRÉ-TEXTUAIS
% ----------------------------------------------------------
% \pretextual

% ---
% Capa
% ---
\imprimircapa
% ---

% ---
% Folha de rosto
% ---
\imprimirfolhaderosto
% ---

% ---
% NOTA DA ABNT NBR 15287:2011, p. 4:
%  ``Se exigido pela entidade, apresentar os dados curriculares do autor em
%     folha ou página distinta após a folha de rosto.''
% ---

% ---
% inserir o sumario
% ---
\tableofcontents
\cleardoublepage
% ---


% ----------------------------------------------------------
% ELEMENTOS TEXTUAIS
% ----------------------------------------------------------
\textual

% ----------------------------------------------------------
% Introdução
% ----------------------------------------------------------
\chapter{Introdução}

Os crescentes avanços tecnológicos que têm ajudado a vida do ser humano também trouxeram diversos novos tipos de crimes relacionados a tecnologia da informação. (citação)
Quando do evento de um crime, há a necessidade de preservar as evidências de modo que apoiem processos legais e investigativos mas também para que possamos compreender a sequência
de eventos envolvidos no ataque de modo que possamos evitar que estas vulnerabilidades sejam exploradas novamente.

\par

Ciência forense digital é uma ferramenta importante para a solução de crimes cometidos com um computador pela sua capacidade de reconstruir a cena do crime a partir das evidências
digitais deixadas por um ataque. A forense digital evoluiu nos últimos anos de uma fase ad hoc, passando por uma fase de adaptação aos requisitos legais até a era de ferramental
avançado de coleta e análise (citação). Com o crescente uso de ferramentas online, virtualização e hospedagem em nuvem, o volume de dados gerado por esses sistemas que já era uma preocupação
da comunidade forense em 2008 (citação), hoje começou a inviabilizar algumas de suas práticas (citação).  

\par

Coma funcionalidade da elasticidade de carga ofertada pelos provedores de nuvem, por meio da qual infraestrutura pode ser alocada e desalocada dinamicamente, veio à tona o problema
da volatilidade dos dados nas máquinas virtuais. Com algumas ameaças que não deixam evidências em disco(citação), a memória de uma máquina despejada de um pool seria para sempre
perdida. Com a máquina mais poderosa hoje ofertada pela Amazon tendo 240Gb de memória (citação), o ferramental forense disponível hoje esta pouco adaptado a desafios trazidos 
pela núvem (citação). A forense digital precisa evoluir.

% ----------------------------------------------------------
% Capitulo de justificativa 
% ----------------------------------------------------------
\chapter{Justificativa}

A tecnologia hoje é parte de nossas vidas e também é parte da vida dos criminosos. O último crime registrado na America do Norte que não envolveu uma evidência digital data de 2011
(citação), no Brasil só em 2015 houve um aumento de 87\% no registro de notas relacionadas a abusos ou crimes digitais (citação).

\par

Do ponto de vista forense, praticantes e pesquisadores concordam que aspectos legais e o crescente volume de dados figuram entre as principais dificuldades trazidas pela computação 
em nuvem (citação). As técnicas forenses atuais são da época pré-nuvem (citação), no momento da coleta de evidências, remove-se o hardware relacionado ao caso e o coloca sob 
custódia. Alguns aspectos da nuvem estão inviabilizando algumas praticas forenses. O aspecto multi inquilino por exemplo impede a remoção do hardware pois o mesmo é compartilhado
com outros usários, removê-la seria uma violação da privacidade de usuários não relacionados a investigação. Outro aspecto da implementação da nuvem é o carater distribuído, 
a evidência pode estar localizada em um país diferente daquele em que o crime foi cometido dificultando a obtenção da mesma (citação). Por último, o crescente volume de dados 
das aplicações atuais deram aos investigadores 6 a 12 meses de backlog de dados para investigar (citação), download de terabytes de dados para análise tem levado horas para ser 
realizado e requer colaboração dos operadores de nuvem (citação). Mudar o paradigma de coleta de evidência tem sido proposto por pesquisadores e praticantes há alguns anos (citação)
, porém esta prática precisa submeter a legislação para que seja aceita em um processo legal.

\par

Por fim, a criatividade dos criminosos para explorar vulnerabilidades trouxe a tona ataques que não deixam evidências em disco, toda a operação ocorre em memória (citação). Com o caráter 
elástico das soluções em nuvem, uma maquina despejada de um pool e cujos recursos foram reutilizados não pode mais prover as evidências necessárias para investigação criminal e 
assim o criminoso fica impune. Coletar a memória de processos de máquinas em núvem de modo que sejam usadas com sucesso em um processo penal, sem sobrecarregar ainda mais os investigadores 
é a principal justificativa desta pesquisa.

\par

Uma máquina virtual é hospedada em um servidor físico real. A coleta de dados não pode ser realizado diretamente de dentro da máquina virtual pois o próprio processo de leitura da
memória a altera. A principal técnica de coleta de informações de memória é a Virtual Machine Introspection (citacao), nesta técnica a máquina é pausada e via hypervisor realiza-se a leitura
da memória e posteriormente uma tradução dos endereços de memória virtual para o real. As desvantagens desta implementação estão na necessidade de tradução dos endereços de memória,
para realização da análise e necessidade de credenciais pra acesso e integração com o hypervisor. (citacao). Na frente de volume de dados baixados temos as técnicas da \"Proof of
retrieveability\" e \"Proof of Data Possession\" que se apoiam na transmissão e comparação de hashes gerados a partir do conteúdo que se quer coletar porém o uso destas requer
quer a informação tenha baixa volatilidade o que não é o caso da informação contida em memória temporária. Na frente legal, as abordagens atuais de live forensics onde esta pesquisa
se concentra carecem de credibilidade por não produzirem evidências aceitas em um processo legal.(citação).

% ----------------------------------------------------------
% Capitulo de Objetivos 
% ----------------------------------------------------------
\chapter{Objetivos}

\begin{itemize}
 \item Objetivo Principal

 \begin{itemize}
  \item O objetivo principal desta pesquisa é derivar um arcabouço que permita coletar a informação da memória de processos de máquinas em nuvem de modo que seja aceita em um processo legal,
 sem violar jurisdição internacional e a privacidade dos usuários de nuvem não relacionados a investigação. Com isso confirmando as hipóteses relacionadas a este trabalho.
 
 \begin{itemize}
  \item Hipotese 1: É possível coletar dados de memória de processos em máquinas na nuvem de modo que se submetam com sucesso ao princípio de Daubert.
  \item Hipotese 2: É possível realizar a coleta sem violar a privacidade de usuários da nuvem não envovidos no caso investigado.
  \item Hipotese 3: É possível coletar tais dados sem envolver jurisdições de outros países.
 \end{itemize}
 
 \end{itemize}
 
\end{itemize}

\begin{itemize}
 \item Sub Objetivos

 \begin{itemize}
  \item Coletar apenas a memória das máquinas involvidas na investigação
  \item Levantar a taxa de erros do processo de coleta
  \item Gerar evidências que se submetam com sucesso ao princípio de Daubert
 \end{itemize}
 
\end{itemize}


% ----------------------------------------------------------
% Plano de Trabalho 
% ----------------------------------------------------------
\chapter{Plano de Trabalho}

\begin{itemize}
 \item reescrever isso
 
 \begin{itemize}
  \item implementação das imagens
  \item implementação dos containers
  \item verificar viabilidade do relacionamento imagem-container-máquina
  \item gerar primeiro lote de evidências
  \item buscar opiniões jurídias sobre a evidência gerada
  \item buscar \% de erro na geração da evidência.
 \end{itemize}

\end{itemize}

% ----------------------------------------------------------
% Material e métodos
% ----------------------------------------------------------
\chapter{Material e Métodos}

\begin{itemize}
 \item reescrever isso
 
 \begin{itemize}
  \item Implementação de virtualização de mercado.
  \item Rede e laboratório para salvar informações coletadas.
  \item Conta no dockerhub
  \item laptop para [especificação aqui] desenvolvimento da solução
 \end{itemize}

\end{itemize}

% ----------------------------------------------------------
% Análise dos resultados
% ----------------------------------------------------------
\chapter{Análise dos Resultados}

\begin{itemize}
 \item reescrever isso
 
 \begin{itemize}
  \item avaliando as opiniões sobre a validade da evidência por representantes legais brasileiros.
  \item verificando se a evidência se encaixa no principio de Daubert.
  \item repetir o processo de coleta e verificar o nível de erro.
 \end{itemize}

\end{itemize}


% ---
% Finaliza a parte no bookmark do PDF
% para que se inicie o bookmark na raiz
% e adiciona espaço de parte no Sumário
% ---
\phantompart

% ----------------------------------------------------------
% Referências bibliográficas
% ----------------------------------------------------------
\bibliography{abntex2-modelo-references}

%---------------------------------------------------------------------
% INDICE REMISSIVO
%---------------------------------------------------------------------

\phantompart

\printindex


\end{document}