%\title{Modelo de Projeto de pesquisa}
%% abtex2-modelo-projeto-pesquisa.tex, v-1.9 laurocesar
%% Copyright 2012-2013 by abnTeX2 group at http://abntex2.googlecode.com/ 
%%
%% This work may be distributed and/or modified under the
%% conditions of the LaTeX Project Public License, either version 1.3
%% of this license or (at your option) any later version.
%% The latest version of this license is in
%%   http://www.latex-project.org/lppl.txt
%% and version 1.3 or later is part of all distributions of LaTeX
%% version 2005/12/01 or later.
%%
%% This work has the LPPL maintenance status `maintained'.
%% 
%% The Current Maintainer of this work is the abnTeX2 team, led
%% by Lauro César Araujo. Further information are available on 
%% http://abntex2.googlecode.com/
%%
%% This work consists of the files abntex2-modelo-projeto-pesquisa.tex
%% and abntex2-modelo-references.bib
%%

% ------------------------------------------------------------------------
% ------------------------------------------------------------------------
% abnTeX2: Modelo de Projeto de pesquisa em conformidade com 
% ABNT NBR 15287:2011 Informação e documentação - Projeto de pesquisa -
% Apresentação 
% ------------------------------------------------------------------------ 
% ------------------------------------------------------------------------

\documentclass[
	% -- opções da classe memoir --
	12pt,				% tamanho da fonte
	openright,			% capítulos começam em pág ímpar (insere página vazia caso preciso)
	oneside,			% para impressão em verso e anverso. Oposto a oneside
	a4paper,			% tamanho do papel. 
	% -- opções da classe abntex2 --
	%chapter=TITLE,		% títulos de capítulos convertidos em letras maiúsculas
	%section=TITLE,		% títulos de seções convertidos em letras maiúsculas
	%subsection=TITLE,	% títulos de subseções convertidos em letras maiúsculas
	%subsubsection=TITLE,% títulos de subsubseções convertidos em letras maiúsculas
	% -- opções do pacote babel --
	english,			% idioma adicional para hifenização
	french,				% idioma adicional para hifenização
	spanish,			% idioma adicional para hifenização
	brazil,				% o último idioma é o principal do documento
	]{abntex2}

% ---
% PACOTES
% ---

% ---
% Pacotes fundamentais 
% ---
\usepackage{lmodern}			% Usa a fonte Latin Modern
\usepackage[T1]{fontenc}		% Selecao de codigos de fonte.
\usepackage[utf8]{inputenc}		% Codificacao do documento (conversão automática dos acentos)
\usepackage{indentfirst}		% Indenta o primeiro parágrafo de cada seção.
\usepackage{color}				% Controle das cores
\usepackage{graphicx}			% Inclusão de gráficos
\usepackage{microtype} 			% para melhorias de justificação
% ---

% ---
% Pacotes adicionais, usados apenas no âmbito do Modelo Canônico do abnteX2
% ---
\usepackage{lipsum}				% para geração de dummy text
% ---

% ---
% Pacotes de citações
% ---
\usepackage[brazilian,hyperpageref]{backref}	 % Paginas com as citações na bibl
\usepackage[alf]{abntex2cite}	% Citações padrão ABNT

% --- 
% CONFIGURAÇÕES DE PACOTES
% --- 

% ---
% Configurações do pacote backref
% Usado sem a opção hyperpageref de backref
\renewcommand{\backrefpagesname}{Citado na(s) página(s):~}
% Texto padrão antes do número das páginas
\renewcommand{\backref}{}
% Define os textos da citação
\renewcommand*{\backrefalt}[4]{
	\ifcase #1 %
		Nenhuma citação no texto.%
	\or
		Citado na página #2.%
	\else
		Citado #1 vezes nas páginas #2.%
	\fi}%
% ---

% ---
% Informações de dados para CAPA e FOLHA DE ROSTO
% ---
\titulo{Coletando dados de memória de uma máquina em nuvem para análise forense}
\autor{Hamilton Fonte II}
\orientador{Marcos Antonio Simplício Jr}
\local{São Paulo, Brasil}
\data{2016, v-0.1}
\instituicao{%
  Universidade de São Paulo -- USP
  \par
  Escola Politécnica - Engenharia de Computação
  \par
  Programa de Pós Graduação em Engenharia Elétrica - Mestrado}
\tipotrabalho{Plano de Pesquisa de Pós-Graduação - Mestrado}
% O preambulo deve conter o tipo do trabalho, o objetivo, 
% o nome da instituição e a área de concentração 
\preambulo{Projeto de pesquisa para a disciplina Metodolodia de Pesquisa 
Científica em Engenharia de Computação.}
% ---

% ---
% Configurações de aparência do PDF final

% alterando o aspecto da cor azul
\definecolor{blue}{RGB}{41,5,195}

% informações do PDF
\makeatletter
\hypersetup{
     	%pagebackref=true,
		pdftitle={\@title}, 
		pdfauthor={\@author},
    	pdfsubject={\imprimirpreambulo},
	    pdfcreator={LaTeX with abnTeX2},
		pdfkeywords={abnt}{latex}{abntex}{abntex2}{projeto de pesquisa}, 
		colorlinks=true,       		% false: boxed links; true: colored links
    	linkcolor=blue,          	% color of internal links
    	citecolor=blue,        		% color of links to bibliography
    	filecolor=magenta,      		% color of file links
		urlcolor=blue,
		bookmarksdepth=4
}
\makeatother
% --- 

% --- 
% Espaçamentos entre linhas e parágrafos 
% --- 

% O tamanho do parágrafo é dado por:
\setlength{\parindent}{1.3cm}

% Controle do espaçamento entre um parágrafo e outro:
\setlength{\parskip}{0.2cm}  % tente também \onelineskip

% ---
% compila o indice
% ---
\makeindex
% ---

% ----
% Início do documento
% ----
\begin{document}

% Retira espaço extra obsoleto entre as frases.
\frenchspacing 

% ----------------------------------------------------------
% ELEMENTOS PRÉ-TEXTUAIS
% ----------------------------------------------------------
% \pretextual

% ---
% Capa
% ---
\imprimircapa
% ---

% ---
% Folha de rosto
% ---
\imprimirfolhaderosto
% ---

% ---
% NOTA DA ABNT NBR 15287:2011, p. 4:
%  ``Se exigido pela entidade, apresentar os dados curriculares do autor em
%     folha ou página distinta após a folha de rosto.''
% ---

% ---
% inserir o sumario
% ---
\tableofcontents
\cleardoublepage
% ---


% ----------------------------------------------------------
% ELEMENTOS TEXTUAIS
% ----------------------------------------------------------
\textual

% ----------------------------------------------------------
% Introdução
% ----------------------------------------------------------
\chapter{Introdução}

Aqui vai a introdução

% ----------------------------------------------------------
% Capitulo de justificativa 
% ----------------------------------------------------------
\chapter{Justificativa}

Aqui vai a justificativa

% ----------------------------------------------------------
% Capitulo de Objetivos 
% ----------------------------------------------------------
\chapter{Objetivos}

Aqui vão os objetivos

% ----------------------------------------------------------
% Plano de Trabalho 
% ----------------------------------------------------------
\chapter{Métodos}

Aqui vão os métodos

% ----------------------------------------------------------
% Material e métodos
% ----------------------------------------------------------
\chapter{Revisão Bibliográfica}

\begin{itemize}

\item \textbf{Digital forensics framework for a cloud environment \cite{George2012} }: Framework para coleta de dados de máquinas virtuais. Tem duas formas de acionamento,
a manual e outra integrada com algum sistema de detecção de ameaça. Quando acionado, escuta a rede, determina qual a máquina é objeto de investigação e coleta informações de 
log e tráfego de rede e associa ao usuário das respectivas máquinas. Propõe o armazenamento das evidências em local fora da núvem para escapar de problemas de jurisdição e
multi-inquilino mas tem inteligência para usar a própria nuvem como armazenamento caso o espaço na midia extena acabe.

O framework proposto, apesar de cobrir o cenário em que uma máquina é despejada do pool e os recursos liberados, não dá detalhes o suficiente para demonstrar que consegue
reproduzir o processo e obter o mesmo resultado. Dá a entender que é aplicável apenas a um sistema virtual estático, onde o número e organização das máquinas é constante. 
De informação volátil, coleta apenas tráfego de rede, não coleta memória. Com a forma de acionamento descrito ele não consegue descrever, com as evidências, como era o 
sistema antes do ataque. Apesar de armazenar a evidëncia fora da nuvem, não dá detalhes de cadeia de custódia, garantia de integridade e confidencialidade. Esta abordagem 
tem um viés mais técnico, leva poucas questões jurídicas em consideração.

De melhor, eu proponho autilização de container para associar a evidência a sua origem tornando o processo independente de máquina. Como eu implemento uma janela de x dias
de coleta antes da detecção do ataque eu consigo descrever, através de evidência como era o sistema antes do mesmo. Com isso consigo evidências em um cenário
de infra-estrutura dinâmica. Eu tomo precauções para garantir a integridade e confidencialidade dos dados transportando via TLS os dados para um local fora da nuvem e 
estabelecendo um controle de acesso.\\
 
\item \textbf{Evidence and cloud computing the virtual machine introspection approach \cite{Poisel2013} }: Descreve um método de coleta de informações de máquinas
em nuvem através da técnica de introspecção em máquina virtual, onde se acessa os dados das máquinas virtuais através do hypervisor. Propõe que o processo seja disparado
por demanda atrelado a um sistema de detecção de ameaça ou intervenção humana.

A técnica descrita cobre apenas o processo de coleta de informações, não explica onde ou como elas serão armazenadas. No que tange as informações de memória, como os 
endereços de memória são os do host, estes precisam ser traduzidos para que a análise forense seja feita. Segundo a comunidade, tal estratégia é imune a técnicas anti-
forenses empregadas por usuários maliciosos pois esta localizada fora da VM. Como a abordagem não tem conhecimento do que esta rodando dentro da máquina virtual precisa
de uma copia bit a bit da evidência. Embora pareça possível, não descreve como lida com o cenário onde uma máquina é despejada do pool e os recursos liberados. 

De melhor, eu proponho um arcabouço para coleta e armazenamento de evidências, não apenas a coleta. Minha abordagem usa uma estratégia diferente pois coleto a memória diretamente
dentro da máquina virtual. Nessa estratégia eu evito o problema do semantic gap, não preciso realizar tradução de endereços para viabilizar a análise forense mas, de acordo com 
a comunidade fico mais sucetível a técnicas anti-forenses. Minha abordagem teria também a vantagem de conhecer o que esta rodando dentro da máquina e assim formatar a coleta 
de acordo.\\

\item \textbf{Design and implementation of FROST: FoRensic tools for Open STack \cite{Dykstra2013} }: Framework para coleta de dados de máquinas virtuais através da API do
hypervisor. Isola a máquina virtual afetada do pool original para realização da coleta. Precisa ser acionado quando uma ameaça é detectada. É o mais bem acabado arcabouço de
todas as propostas encontradas até agora mas ainda assim é uma proposta técnica, faz breve mensão a aspectos legais. Por estar integrado ao Open Stack o arcabouço depende de
cooperação do provedor de serviços de núvem onde ele esta rodando, isso é considerado problemático pela comunidade pois a prioridade do mesmo é manter o serviço funcionando
e não coletar evidencias forenses. Como roda no hypervisor não conhece o que esta rodando dentro da máquina. Depende da existência da máquina virtual para realização da coleta.

De melhor, eu proponho a utilização de container para associar a evidência de memória a sua origem tornando o processo independente de máquina e a janela de x dias antes da 
detecção do ataque para conseguir descrever o sistema antes do mesmo. Minha proposta não depende de cooperação do provedor do serviço de nuvem. Minha abordagem teria também
a vantagem de conhecer o que esta rodando dentro da máquina e assim formatar a coleta de acordo.\\
  
\item \textbf{Automated Forensic Data Acquisition in the Cloud \cite{Reichert2015} }: Propõe um modelo que tira snapshots de máquinas virtuais atrelado a algum mecanismo de
detecção de ameaça baseado no hypervisor. Usa o Google Rapid Response para salvar as informações coletadas fora da núvem de forma a driblar os problemas de multi-jurisdição e 
multi-inquilino. Descreve satisfatóriamente a cadeia de custódia da evidência.

O modelo proposto só começa a coletar evidência após a detecção da ameaça, e toma um snapshot da máquina toda o que já foi julgado pela comunidade como um processo custoso em
termos de espaco em disco, piora o problema do volume de dados aser analisado e pessoalmente acho ariscado, um exemplo de problema é editar um HD virtual que esta atrelado a 
uma máquina virtual da qual se gerou snapshots que pode levar a perda de dados.

De melhor eu proponho apenas a coleta de informações de memória e uso a janela de coleta de x dias antes do ataque para manter sob controle a quantidade de informação que 
precisa ser analisada. Eu proponho autilização de container para associar a evidência a sua origem tornando o processo independente de máquina. \\
 
\item \textbf{A log based approach to make digital forensics easier on cloud computing \cite{Sang2013} }: Método que sugere salvar a informação coletada fora da núvem de modo a driblar
os problemas de multi-inquilinato e multi-jurisdição, usa um mecanismo de hash para garantir a autenticidade e integridade da informação mas não dá detalhes da implementação e 
não descreve como controla o acesso a evidência armazenada. Segundo o próprio autor, o método não funciona em IaaS. Precisa da cooperação do provedor de nuvem pois depende que 
este último decida quais informações serão adicionadas ao log. O método não é aplicável a coleta de informações de memória.

A proposta não coleta dados de memória por decisão do autor, esta proposta entrou na lista pela abordagem baseada em log. Neste quesito, minha proposta é melhor no que 
garante todos os pontos da cadeia de custódia, integridade e confidencialidade. No âmbito da informação coletada a eu não dependo do provedor de serviço de nuvem para 
conseguir a evidência, o serviço rodaria dentro da máquina. \\

\item \textbf{Volatile memory acquisition using backup for forensic investigation \cite{Dezfouli2012} }: Técnica desenvolvida para dispositivos móveis que sugere a utilização
do próprio como repositório das evidências coletadas da memória. Para manter a utilização de espaço ao mínimo sugere manter apenas o último estado conhecido da memória.\\
 
É uma técnica interessante do ponto de vista de estratégia de armazenamento quando guarda apenas o último estado da memória. Essa abordagem porém perde a informação do momento 
do ataque e não consegue descrever o sistema antes do mesmo. Do resto da proposta não é aplicável para este projeto pois, armazenando a evidência na máquina a mesma seria perdida
quando a máquina fosse despejada do pool e seus recursos liberados. Outro aspecto é a cadeia de custódia que não é abordada na proposta.
 
 \end{itemize}
 
% ----------------------------------------------------------
% Análise dos resultados
% ----------------------------------------------------------
\chapter{Análise dos Resultados}

Aqui a análise dos resultados

% ---
% Finaliza a parte no bookmark do PDF
% para que se inicie o bookmark na raiz
% e adiciona espaço de parte no Sumário
% ---
\phantompart

% ----------------------------------------------------------
% ELEMENTOS PÓS-TEXTUAIS
% ----------------------------------------------------------
\postextual

% ----------------------------------------------------------
% Referências bibliográficas
% ----------------------------------------------------------
\bibliography{abntex2-modelo-references}

%---------------------------------------------------------------------
% INDICE REMISSIVO
%---------------------------------------------------------------------

\phantompart

\printindex


\end{document}
